%%%%%%%%%%%%%%%%%%%%%%%%%%%%%%%%%%%%%%%%%%%%%%%%%%%%%%%%%%%%%%%%%%%%%%
% LaTeX Example: Project Report
%
% Source: http://www.howtotex.com
%
% Feel free to distribute this example, but please keep the referral
% to howtotex.com
% Date: March 2011 
% 
%%%%%%%%%%%%%%%%%%%%%%%%%%%%%%%%%%%%%%%%%%%%%%%%%%%%%%%%%%%%%%%%%%%%%%

\documentclass[paper=letter, fontsize=11pt]{article}
\usepackage[T1]{fontenc}
\usepackage{fourier}

\usepackage[english]{babel}															% English language/hyphenation
%\usepackage[protrusion=true,expansion=true]{microtype}	
\usepackage{amsmath,amsfonts,amsthm} % Math packages
\usepackage[pdftex]{graphicx}	
\usepackage{url}
\usepackage{siunitx}
\usepackage{subfig}
\usepackage{pgf}
\usepackage{float}



%%% Custom sectioning
\usepackage{sectsty}
\allsectionsfont{ \normalfont\scshape}


%%% Custom headers/footers (fancyhdr package)
\usepackage{fancyhdr}
\pagestyle{fancyplain}
\fancyhead{}											% No page header
\fancyfoot[L]{}											% Empty 
\fancyfoot[C]{}											% Empty
\fancyfoot[R]{\thepage}									% Pagenumbering
\renewcommand{\headrulewidth}{0pt}			% Remove header underlines
\renewcommand{\footrulewidth}{0pt}				% Remove footer underlines
\setlength{\headheight}{13.6pt}


%%% Equation and float numbering
\numberwithin{equation}{section}		% Equationnumbering: section.eq#
\numberwithin{figure}{section}			% Figurenumbering: section.fig#
\numberwithin{table}{section}				% Tablenumbering: section.tab#


%%% Maketitle metadata
\newcommand{\horrule}[1]{\rule{\linewidth}{#1}} 	% Horizontal rule

\title{
		%\vspace{-1in} 	
		\usefont{OT1}{bch}{b}{n}
		\normalfont \normalsize \textsc{State University at Buffalo} \\ [25pt]
		\horrule{0.5pt} \\[0.4cm]
		\Large Project 3 : Classification Algorithms \\
		\horrule{2pt} \\[0.5cm]
}
\author{
		\normalfont\large 								
        Yuze Liu \hspace{0.6cm}50207903\\
        \normalfont\large 
        Luting Chen \hspace{0.5cm}50133507\\
        \normalfont\large 
        Vicky Zheng \hspace{0.5cm}50037709\\
}
\date{\large 12/10/2016}
%%% Begin document
\begin{document}
\maketitle
%\newdimen\origiwspc%
\section{K-Nearest Neighbors Algorithm}
\subsection{Algorithm Description}

\subsection{Pros and Cons}

\subsection{Result Evaluation}

\section{Decision Tree}
\subsection{Algorithm Description}

\subsection{Pros and Cons}

\subsection{Result Evaluation}

\section{Decision Tree with Random Forest}
\subsection{Algorithm Description}
\subsection{Pros and Cons}
\subsection{Result Evaluation}

\section{Decision Tree with Boosting}
\subsection{Algorithm Description}
\subsection{Pros and Cons}

\subsection{Result Evaluation}

\section{Naive Bayes}
\subsection{Algorithm Description}
Naive bayes is a classification algorithm that is based on the Bayes Theorem. Naive bayes aims to classify the probability of a sample $X$ being in a class $H_i$ using Bayes Theorem. The Bayes theorem is:\\
\begin{center} $P(H_i|X) = \frac{P(H_i)P(X|H_i)}{P(X)}$ \end{center}

\noindent The things that we are classifying often have multiple attributes which we will refer to as $A$ where $A = (A_1,A_2,...,A_d)$. Let $X$ be something we are trying to classify and $X = (x_1, x_2, ..., x_d)$. $P(X)$ is the prior probability of $X$ where $P(x_j) = \frac{n_j}{n}$ where $n_j$ is the number of of training samples in our training set that have the value $x_j$ for attribute $A_j$. 

% The probability that a given sample $X$ is in a class $H_i$ is: \begin{center} $P(H_i|X) = P(H_i|x_1)*P(H_i|x_2)*...*P(H_i|x_d)*P(H_i)$ \end{center}
$P(H_i)$ is simply the class prior probability. 

$P(X|H_i)$,the descriptor posterior probability, can be calculated by:

\begin{center} $P(X|H_i) = \prod_{i=1}^{d} P(x_j, H_i)$ \end{center}

\noindent Calculating the descriptor posterior probability reveals on of the weaknesses of Naive Bayes. If a single value $P(x_j, H_i)$ is 0 then the entire product of $P(X|H_i) = \prod_{i=1}^{d} P(x_j, H_i)$ will evaulate to 0 which will cause $P(H_i|X) = \frac{P(H_i)P(X|H_i)}{P(X)}$ to also evaluate to 0. This can be corrected by using a Laplacian correction where if there is an $n_j = 0$, then we will simply add 1 to every $n_j$ and increase the total number of samples to $n+k$ where $k$ is the number of possible values the attribute $A_j$ can take on. 

\noindent You will also notice that $P(H_i|X) = \frac{P(H_i)P(X|H_i)}{P(X)}$ does not work for continuous values because we cannot count continuous values to get posterior probabilities. I chose to address this by assuming a Gaussian distribution to use:

\begin{center} $P(H_i | x_j) = \frac{1}{\sqrt{2\pi\sigma_{H_i,x_j}^2}}e^{-\frac{1}{2}(\frac{H_i-\mu_{H_i,x_j}}{\sigma_{H_i,x_j}}^2)}$ \end{center}

\noindent I chose to use this method because it seemed to be one of the most popular methods out there for dealing with continuous values when implementing Naive Bayes. 

\subsection{Pros and Cons}
Some of the pros of using Naive bayes is that its simple to implement and it is efficent. One of the cons, however, is that it assumes attribute independence, which of course is not true for all datasets. Another con is that the descriptor posterior probability may evaluate to 0 - we have to prevent this by using a Laplacian correction. This can happen often in small datasets so Naive bayes performs best on large datasets. 

\noindent Something else that may be considered a con is that there are multiple ways to deal with continuous attributes. I chose to handle this by using Gaussian Naive Bayes. This may not perform well for other datasets that have different distributions. 

\subsection{Result Evaluation}
I implemented k-cross fold by randomly shuffling my dataset, and partitioning it into $k$ test sets and $k$ training sets. I took the average of the $k$ accuracy, precision, recall and F-values. \\

\noindent \underline{Results for project3 dataset1.txt are}: \\ 
\textbf{Accuracy}: 0.712582417582\\
\textbf{Precision}: 0.589052375343\\
\textbf{Recall}: 0.748140191881\\
\textbf{F}: 0.657284240816\\

\noindent \underline{Results for project3 dataset2.txt are}:\\
\textbf{Accuracy}: 0.671195652174\\
\textbf{Precision}: 0.51574025974\\
\textbf{Recall}: 0.761451858741\\
\textbf{F}: 0.612021438585\\

\begin{thebibliography}{1}
\bibitem{nv_bayes}
Gaussian Example: https://www.youtube.com/watch?v=r1in0YNetG8
\end{thebibliography}


\end{document}