%%%%%%%%%%%%%%%%%%%%%%%%%%%%%%%%%%%%%%%%%%%%%%%%%%%%%%%%%%%%%%%%%%%%%%
% LaTeX Example: Project Report
%
% Source: http://www.howtotex.com
%
% Feel free to distribute this example, but please keep the referral
% to howtotex.com
% Date: March 2011 
% 
%%%%%%%%%%%%%%%%%%%%%%%%%%%%%%%%%%%%%%%%%%%%%%%%%%%%%%%%%%%%%%%%%%%%%%

\documentclass[paper=letter, fontsize=11pt]{article}
\usepackage[T1]{fontenc}
\usepackage{fourier}

\usepackage[english]{babel}															% English language/hyphenation
%\usepackage[protrusion=true,expansion=true]{microtype}	
\usepackage{amsmath,amsfonts,amsthm} % Math packages
\usepackage[pdftex]{graphicx}	
\usepackage{url}
\usepackage{siunitx}
\usepackage{subfig}
\usepackage{pgf}
\usepackage{float}



%%% Custom sectioning
\usepackage{sectsty}
\allsectionsfont{ \normalfont\scshape}


%%% Custom headers/footers (fancyhdr package)
\usepackage{fancyhdr}
\pagestyle{fancyplain}
\fancyhead{}											% No page header
\fancyfoot[L]{}											% Empty 
\fancyfoot[C]{}											% Empty
\fancyfoot[R]{\thepage}									% Pagenumbering
\renewcommand{\headrulewidth}{0pt}			% Remove header underlines
\renewcommand{\footrulewidth}{0pt}				% Remove footer underlines
\setlength{\headheight}{13.6pt}


%%% Equation and float numbering
\numberwithin{equation}{section}		% Equationnumbering: section.eq#
\numberwithin{figure}{section}			% Figurenumbering: section.fig#
\numberwithin{table}{section}				% Tablenumbering: section.tab#


%%% Maketitle metadata
\newcommand{\horrule}[1]{\rule{\linewidth}{#1}} 	% Horizontal rule

\title{
		%\vspace{-1in} 	
		\usefont{OT1}{bch}{b}{n}
		\normalfont \normalsize \textsc{State University at Buffalo} \\ [25pt]
		\horrule{0.5pt} \\[0.4cm]
		\Large Project 3 : Classification Algorithms \\
		\horrule{2pt} \\[0.5cm]
}
\author{
		\normalfont\large 								
        Yuze Liu \hspace{0.6cm}50207903\\
        \normalfont\large 
        Luting Chen \hspace{0.5cm}50133507\\
        \normalfont\large 
        Vicky Zheng \hspace{0.5cm}50037709\\
}
\date{\large 12/10/2016}
%%% Begin document
\begin{document}
\maketitle
%\newdimen\origiwspc%
\section{K-Nearest Neighbors Algorithm}
\subsection{Algorithm Description}

\subsection{Implementation}

\subsection{Result Evaluation}

\section{Decision Tree}
\subsection{Algorithm Description}

\subsection{Implementation}

\subsection{Result Evaluation}

\section{Decision Tree with Random Forest}
\subsection{Algorithm Description}

\subsection{Implementation}

\subsection{Result Evaluation}

\section{Decision Tree with Boosting}
\subsection{Algorithm Description}

\subsection{Implementation}

\subsection{Result Evaluation}

\section{Naive Bayes}
\subsection{Algorithm Description}

\subsection{Implementation}

\subsection{Result Evaluation}

% \begin{thebibliography}{1}
% \bibitem{sp_clustering}
% Ng, Andrew Y., Michael I. Jordan, and Yair Weiss. "On spectral clustering: Analysis and an algorithm." Advances in neural information processing systems 2 (2002): 849-856.
% \end{thebibliography}


\end{document}