% Use only LaTeX2e, calling the article.cls class and 12-point type.

\documentclass[12pt]{article}

% Users of the {thebibliography} environment or BibTeX should use the
% scicite.sty package, downloadable from *Science* at
% www.sciencemag.org/about/authors/prep/TeX_help/ .
% This package should properly format in-text
% reference calls and reference-list numbers.

%\usepackage{scicite}

% Use times if you have the font installed; otherwise, comment out the
% following line.

\usepackage{times}
\usepackage{graphicx}
\usepackage{lettrine}
% The preamble here sets up a lot of new/revised commands and
% environments.  It's annoying, but please do *not* try to strip these
% out into a separate .sty file (which could lead to the loss of some
% information when we convert the file to other formats).  Instead, keep
% them in the preamble of your main LaTeX source file.


% The following parameters seem to provide a reasonable page setup.

\topmargin 0.0cm
\oddsidemargin 0.2cm
\textwidth 16cm 
\textheight 21cm
\footskip 1.0cm


%The next command sets up an environment for the abstract to your paper.

\newenvironment{sciabstract}{%
\begin{quote} \bf}
{\end{quote}}


% If your reference list includes text notes as well as references,
% include the following line; otherwise, comment it out.

\renewcommand\refname{References and Notes}

% The following lines set up an environment for the last note in the
% reference list, which commonly includes acknowledgments of funding,
% help, etc.  It's intended for users of BibTeX or the {thebibliography}
% environment.  Users who are hand-coding their references at the end
% using a list environment such as {enumerate} can simply add another
% item at the end, and it will be numbered automatically.

\newcounter{lastnote}
\newenvironment{scilastnote}{%
\setcounter{lastnote}{\value{enumiv}}%
\addtocounter{lastnote}{+1}%
\begin{list}%
{\arabic{lastnote}.}
{\setlength{\leftmargin}{.22in}}
{\setlength{\labelsep}{.5em}}}
{\end{list}}


% Include your paper's title here

\title{ Apriori implementation using Matlab} 


% Place the author information here.  Please hand-code the contact
% information and notecalls; do *not* use \footnote commands.  Let the
% author contact information appear immediately below the author names
% as shown.  We would also prefer that you don't change the type-size
% settings shown here.

\author
{Luting Chen  50133507,
	Yuze Liu 50207903, 
	Vicky Zheng  50037709 \\
%\normalsize{$^{1}$Department of Chemistry, University of Wherever,}\\
%\normalsize{An Unknown Address, Wherever, ST 00000, USA}\\
%\normalsize{$^{2}$Another Unknown Address, Palookaville, ST 99999, USA}\\
\\
%\normalsize{$^\ast$To whom correspondence should be addressed; E-mail:  jsmith@wherever.edu.}
}

% Include the date command, but leave its argument blank.

\date{}
%%%%%%%%%%%%%%%%% END OF PREAMBLE %%%%%%%%%%%%%%%%
\begin{document} 

% Double-space the manuscript.

\baselineskip24pt

% Make the title.

\maketitle 
% Place your abstract within the special {sciabstract} environment.

%\begin{sciabstract}
 % This document presents a number of hints about how to set up your
  %{\it Science\/} paper in \LaTeX\ .  We provide a template file,
  %\texttt{scifile.tex}, that you can use to set up the \LaTeX\ source
  %for your article.  An example of the style is the special
  %\texttt{\{sciabstract\}} environment used to set up the abstract you
  %see here.
%\end{sciabstract}
% In setting up this template for *Science* papers, we've used both
% the \section* command and the \paragraph* command for topical
% divisions.  Which you use will of course depend on the type of paper
% you're writing.  Review Articles tend to have displayed headings, for
% which \section* is more appropriate; Research Articles, when they have
% formal topical divisions at all, tend to signal them with bold text
% that runs into the paragraph, for which \paragraph* is the right
% choice.  Either way, use the asterisk (*) modifier, as shown, to
% suppress numbering.

\section*{Introduction}
The apriori algorithm is used to determine association rules. In this homework, we implemented the apriori algorithm by using Matlab. The dataset we used the algorithm on is gene data. The applications of this are straight forward: with association rules we can see how genes relate to each other and what the relation between genes can tell us about the data. 

\section*{Data Preprocesing}
\noindent The data provided is a .txt file that contains a 100 by 102 matrix. There are 100 samples each corresponding to the genes of a patient with a disease. The diseases are: ALL, AML, Breast Cancer and Colon Cancer. In order to optimize our program, we converted each of these values into integers: 1,2,3,4 respectively. The data set also gives us a sequence of 100 genes labeled as either up or down. Each gene at index i is treated as 1 if it is up and 0 if it is down. We did this to optimize our program by having fewer string comparsions since integer comparisons are much faster. 

\section*{Implementation}
\noindent For this homework, we choose to use Matlab because our input data is a matrix. Matlab has helpful functions for matrix manipulation like determining whether a set of numbers is in a row of the matrix. It also has helpful functions such as the union of two sets which helped us do the pruning necessary for an efficient apriori algorithm.


\section*{Conclusion}

In conclusion, BioGalaxy can have reasonably timed queries since having multiple fact tables allows for fewer joins. Our schema also supports many to many relationships, temporal data, and hierarchal querying.

% Your references go at the end of the main text, and before the
% figures.  For this document we've used BibTeX, the .bib file
% scibib.bib, and the .bst file Science.bst.  The package scicite.sty
% was included to format the reference numbers according to *Science*
% style.

\bibliography{scibib}

\bibliographystyle{Science}

\begin{thebibliography}{9}
	\bibitem{latexcompanion} 
	Liangjiang Wang, and Aidong Zhang 
	\textit{BioStar Models of clinical and genomic data for biomedical data warehouse design}. 
	
	
	\bibitem{knuthwebsite} 
	Wide Skills
	\\\texttt{$http://www.wideskills.com/data-warehousing/dimensional-data-modelling$}
	
	\bibitem{Wiki Pedia Website} 
	Wiki Pedia Website
	\\\texttt{$https://en.wikipedia.org/wiki/Fact_constellation$}
\end{thebibliography}


\end{document}




















